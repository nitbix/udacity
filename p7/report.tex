\documentclass[12pt,a4paper]{article}

\begin{document}

\title{Udacity, A/B Testing}
\author{Alan Mosca}

\maketitle

\section{Metrics Choice}

Invariant Metrics:

\begin{itemize}
\item Number of Cookies
\item Number of Clicks
\end{itemize}

Evaluation Metrics:

\begin{itemize}
\item Gross Conversion
\item Net Conversion
\item Retention
\end{itemize}

\section{Variability}
Given the following data about the experiment:
\begin{itemize}
\item Unique cookies to view page per day $N_{view} = 40000$
\item Unique cookies to click "Start free trial" per day $N_{click} = 3200$
\item Enrolments per day $E = 660$
\item Click-through probability on "Start free trial" $P(click) = 0.08$
\item $P(E|click) = 0.20625$
\item $P(Pay|E) = 0.53$
\item $P(Pay|Click) = 0.1093125$
\end{itemize}

If we assume a sample size $S = 5000$ unique cookies visiting the page, based on
the original data, the number of clicks would be:
\begin{equation}
\hat{N}_{click} = S * P(click) = 400
\end{equation}
and the number of enrolments:
\begin{equation}
\hat{N}_{enroll} = S * P(click) * P(E|click) = 82.5
\end{equation}


Given that all the chosen metrics can be assumed to follow a Binomial
distribution, the estimate of Standard Error follows:
\begin{equation}
SE = \sqrt{\frac{\hat{p}(1-\hat{p})}{N}}
\end{equation}

so for each metric, the SE is:
\begin{itemize}
\item Gross Conversion: $0.020231$
\item Retention: $0.0549$
\item Net conversion: $0.015602$
\end{itemize}

\section{Sizing}
\subsection{Number of Samples}
With a type I error rate of $\alpha=0.05$ and a type II error of $\beta = 0.2$,
the minimum detectable effects are $d_{gc,min} = 0.01$, $d_{r,min} = 0.01$,
$d_{nc,min} = 0.0075$.
I will not be using the Bonferroni correction, because the measures are
covariant.

The ratios of clicks to pageviews is $\frac{N_{clicks}}{N_{views}} = 0.08$ and the
ration of conversions to pageviews is $\frac{N_{enroll}}{N_{views}} = 0.0165$,
which will be used for corrections.

The non-corrected experiment sizes are as follows:
\begin{itemize}
\item Gross Conversion: $25835$
\item Net Conversion: $27413$
\end{itemize}

After the corrections they are as follows:
\begin{itemize}
\item Gross Conversion: $322938$
\item Net Conversion: $342662.5$
\end{itemize}

Doubling the maximum because we need this number of pageviews for each
hypothesis, we get a total of $685325$ pageviews required.

\subsection{Duration vs Exposure}
Given the relatively low risk nature of the experiment\footnote{It does not
affect any of the pages with content, so it does not affect existing users, and
it is only adding a small prompt that wouldn't be considered annoying by most
people}, it makes sense to divert a large amount of traffic to it to reach a
conclusion quickly. However, it also makes sense to keep a small amount of
traffic out of the experiment, in case there was an error or bug in the
experiment, so that it can be detected quickly.  Therefore I would say that
diverting $90$\% of Udacity traffic makes sense. Give a daily traffic of $40000$
pageviews, this means there would be $36000$ pageviews dedicated to the
experiment. The total duration of the experiment would be of $19.03$ days, which
will be rounded to $20$. This is an acceptable time.

\section{Sanity Checks}
We used a $95$\% confidence interval, with a Z-score of $1.96$. For both
measures we expect a probability of $0.5$.

For the number of cookies, we had a SE of $0.0006$, giving a margin of error of
$0.0012$ and a confidence interval of $[ 0.4988, 0.5012 ]$. The observed value
was $0.5006$, so it is considered a pass.

For the number of clicks, we had a SE of $0.0021$, giving a margin of error of
$0.0041$ and a confidence interval of $[ 0.4958, 0.5041 ]$. The observed value
was $0.5005$, so it is considered a pass.

\section{Effect Size Tests}
The measured values from the experiment are as follows:

\subsection{Gross Conversion}
\begin{itemize}
\item $P(gc|control) = 0.2189$
\item $P(gc|experiment) = 0.1983$
\item $\hat{d} = -0.0206$
\item $SE = 0.0044$
\item $ME = 8.5652 \times 10^{-3}$
\item $CI = [-0.0291,-0.0120]$
\item $d_{min} = 0.01$
\end{itemize}

Gross Conversion is therefore both statistically and practically significant. We
therefore conclude that we have observed a valid change in Gross Conversion.

\subsection{Net Conversion}
\begin{itemize}
\item $P(nc|control) = 0.1176$
\item $P(nc|experiment) = 0.1127$
\item $\hat{d} = -0.0049$
\item $SE = 3.4340 \time 10^{-3}$
\item $ME = 6.7228 \times 10^{-3}$
\item $CI = [-0.0116,0.0018]$
\item $d_{min} = 0.0075$
\end{itemize}

Net conversion is not statistically, or practically, significant. This
effectively means that we do not see a change in Net Conversions.

\section{Sign Tests}

\section{Results Summary}

\section{Recommendation}
The reduction in Gross Conversions is part of what was expected of the new
change. The number of people signing up would be reduced, and there is no
significant change in the Net Conversions. This means that there are less people
signing up and then dropping out. Conversely, the lack of significance on the
Net Conversions also means that there is no significant lost revenue. This means
that there is inherent savings in saved resources and a more focused trial. I
recommend launching the change even though there is no increase in revenue.

\section{Follow-up Experiment}
If the main concern is for people to be able to pass their initial trial, I
would recommend testing


\end{document}
