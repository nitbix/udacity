\documentclass[12pt,a4paper]{article}

\begin{document}

\title{Udacity, A/B Testing}

\maketitle

\section{Metrics Choice}

Invariant Metrics:

\begin{itemize}
\item Number of Cookies
\item Number of Clicks
\end{itemize}

Evaluation Metrics:

\begin{itemize}
\item Gross Conversion
\item Net Conversion
\item Retention
\end{itemize}

\section{Variability}
Given the following data about the experiment:
\begin{itemize}
\item Unique cookies to view page per day $N_{view} = 40000$
\item Unique cookies to click "Start free trial" per day $N_{click} = 3200$
\item Enrolments per day $E = 660$
\item Click-through probability on "Start free trial" $P(click) = 0.08$
\item $P(E|click) = 0.20625$
\item $P(Pay|E) = 0.53$
\item $P(Pay|Click) = 0.1093125$
\end{itemize}

If we assume a sample size $S = 5000$ unique cookies visiting the page, based on
the original data, the number of clicks would be:
\begin{equation}
\hat{N}_{click} = S * P(click) = 400
\end{equation}
and the number of enrolments:
\begin{equation}
\hat{N}_{enroll} = S * P(click) * P(E|click) = 82.5
\end{equation}


Given that all the chosen metrics can be assumed to follow a Binomial
distribution, the estimate of Standard Error follows:
\begin{equation}
SE = \sqrt{\frac{\hat{p}(1-\hat{p})}{N}}
\end{equation}

so for each metric, the SE is:
\begin{itemize}
\item Gross Conversion: $0.020231$
\item Retention: $0.0549$
\item Net conversion: $0.015602$
\end{itemize}

\section{Sizing}
\subsection{Number of Samples}
With a type I error rate of $\alpha=0.05$ and a type II error of $\beta = 0.2$,
the minimum detectable effects are $d_{gc,min} = 0.01$, $d_{r,min} = 0.01$,
$d_{nc,min} = 0.0075$.
I will not be using the Bonferroni correction, because the measures are
covariant.

The ratios of clicks to pageviews is $\frac{N_{clicks}}{N_{views}} = 0.08$ and the
ration of conversions to pageviews is $\frac{N_{enroll}}{N_{views}} = 0.0165$,
which will be used for corrections.

The non-corrected experiment sizes are as follows:
\begin{itemize}
\item Gross Conversion: $25835$
\item Net Conversion: $27413$
\end{itemize}

After the corrections they are as follows:
\begin{itemize}
\item Gross Conversion: $322938$
\item Net Conversion: $342662.5$
\end{itemize}

Doubling the maximum because we need this number of pageviews for each
hypothesis, we get a total of $685325$ pageviews required.

\subsection{Duration vs Exposure}
Given the relatively low risk nature of the experiment\footnote{It does not
affect any of the pages with content, so it does not affect existing users, and
it is only adding a small prompt that wouldn't be considered annoying by most
people}, it makes sense to divert a large amount of traffic to it to reach a
conclusion quickly. However, it also makes sense to keep a small amount of
traffic out of the experiment, in case there was an error or bug in the
experiment, so that it can be detected quickly.  Therefore I would say that
diverting $90$\% of Udacity traffic makes sense. Give a daily traffic of $40000$
pageviews, this means there would be $36000$ pageviews dedicated to the
experiment. The total duration of the experiment would be of $19.03$ days, which
will be rounded to $20$. This is an acceptable time.

\section{Sanity Checks}

\section{Effect Size Tests}

\section{Sign Tests}

\section{Results Summary}

\section{Recommendation}

\section{Follow-up Experiment}


\end{document}
