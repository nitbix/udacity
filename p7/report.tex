\documentclass[12pt,a4paper]{article}

\begin{document}

\title{Udacity, A/B Testing}
\author{Alan Mosca}

\maketitle

\section{Metrics Choice}

\subsection{Invariant Metrics}

\begin{itemize}
\item Number of Cookies
This was chosen because it is an easy way of identifying unique visitors, which
makes it a good choice for a unit of diversion. It has not been chosen as an
evaluation metric because it shouldn't be affected by the change that is being
evaluated.

\item Number of Clicks
This was chosen as an invariant (and not for evaluation) because it also isn't
affected by the change that is being evaluated.
\end{itemize}

\subsection{Evaluation Metrics}

\begin{itemize}
\item Gross Conversion
This measure is directly affected by the experiment, and it helps to determine
if the number of people that click on the sign-up link for the trial and then
leave after it ends can be reduced.

\item Net Conversion
This measure is directly affected by the experiment, and it helps to determine
if the number of people that click on the sign-up link and pay is affected.

\end{itemize}

\subsection{Discarded Metrics}
\begin{itemize}
\item Number of user-ids
This measure is affected by the experiment, as we can't expect the number of
user-ids to be split evenly between control and experiment. It could be an
evaluation metric because it measures the number of students that make it past
the free trial, but it is not normalized by the number of clicks, so we prefer
other measures.

\item Retention
This was initially selected as an evaluation metric because it directly measures
the effect of the experiment, but using it would have made the length of the
experiment $132$ days, which was too high, so it has been subsequently
discarded.  It can't be used as an invariant because the experiment affects this
value.

\item Click-through-probability
This metric is not affected by the experiment, given that the click happens
before the diversion. Nevertheless, it does not tell us anything about the
experiment and is not a unit of diversion, so it is not very useful.

\subsection{Goal}
The goal of the experiment is to determine whether adding the question ``are you
able to dedicate at least $5$ hours per week to your studies'' reduces the
number of people that sign up for a trial and then leave before paying for the
first time. Additionally, we want to know whether adding this question would
prevent people who will eventually go through to payment from signing up.
We can therefore state our null hypothesis $h_0$ to be that adding the question
to the process does not affect the proportion of people signing up for the
trial, or the proportion of people paying wrt the number of people clicking on
the sign-up button. Broken down wrt the two measures, we define:

\begin{itemize}
\item $h_0(gc)$: the Gross Conversion is not affected by the experiment
\item $h_0(nc)$: the Net Conversion is not affected by the experiment
\end{itemize}

In order to accept the change, we therefore want to ideally reject the null
hypothesis on Gross Conversion $h_0(gc)$ with a \emph{negative} value of
$\hat{d}$ and accept the null hypothesis on Net Conversion $h_0(nc)$, or accept
it with a \emph{positive} value of $\hat{d}$ (which means that revenue would
actually increase).

\end{itemize}

\section{Variability}
Given the following data about the experiment:
\begin{itemize}
\item Unique cookies to view page per day $N_{view} = 40000$
\item Unique cookies to click "Start free trial" per day $N_{click} = 3200$
\item Enrolments per day $E = 660$
\item Click-through probability on "Start free trial" $P(click) = 0.08$
\item $P(E|click) = 0.20625$
\item $P(Pay|E) = 0.53$
\item $P(Pay|Click) = 0.1093125$
\end{itemize}

If we assume a sample size $S = 5000$ unique cookies visiting the page, based on
the original data, the number of clicks would be:
\begin{equation}
\hat{N}_{click} = S * P(click) = 400
\end{equation}
and the number of enrolments:
\begin{equation}
\hat{N}_{enroll} = S * P(click) * P(E|click) = 82.5
\end{equation}


Given that all the chosen metrics can be assumed to follow a Binomial
distribution, the estimate of Standard Error follows:
\begin{equation}
SE = \sqrt{\frac{\hat{p}(1-\hat{p})}{N}}
\end{equation}

so for each metric, the SE is:
\begin{itemize}
\item Gross Conversion: $0.020231$
\item Net conversion: $0.015602$
\end{itemize}

Given that Gross Conversion and Net Conversion are both using the unit of
diversion as their denominator, I would expect the analytical variance to match
the empirical one.

\section{Sizing}
\subsection{Number of Samples}
With a type I error rate of $\alpha=0.05$ and a type II error of $\beta = 0.2$,
the minimum detectable effects are $d_{gc,min} = 0.01$, $d_{r,min} = 0.01$,
$d_{nc,min} = 0.0075$.
I will not be using the Bonferroni correction, because the measures are
covariant.

The ratios of clicks to pageviews is $\frac{N_{clicks}}{N_{views}} = 0.08$ and the
ration of conversions to pageviews is $\frac{N_{enroll}}{N_{views}} = 0.0165$,
which will be used for corrections.

The non-corrected experiment sizes are as follows:
\begin{itemize}
\item Gross Conversion: $25835$
\item Net Conversion: $27413$
\end{itemize}

After the corrections they are as follows:
\begin{itemize}
\item Gross Conversion: $322938$
\item Net Conversion: $342662.5$
\end{itemize}

Doubling the maximum because we need this number of pageviews for each
hypothesis, we get a total of $685325$ pageviews required.

\subsection{Duration vs Exposure}
The experiment does not
affect any of the pages with content, so it does not affect existing users, and
it is only adding a small prompt that wouldn't be considered annoying by most
people. It cleary doesn't hurt anyone because the people that would be
discouraged by the pop-up message are not likely to finish the course and
receive a certification, so they would not be missing out. There is also no
sensitive data being used for the experiment.
Therefore, it makes sense to divert a large amount of traffic to it to reach a
conclusion quickly. However, it also makes sense to keep a small amount of
traffic out of the experiment, in case there was an error or bug in the
experiment, so that it can be detected quickly.  Therefore I would say that
diverting $90$\% of Udacity traffic makes sense. Give a daily traffic of $40000$
pageviews, this means there would be $36000$ pageviews dedicated to the
experiment. The total duration of the experiment would be of $19.03$ days, which
will be rounded to $20$. This is an acceptable time.

\section{Sanity Checks}
We used a $95$\% confidence interval, with a Z-score of $1.96$. For both
measures we expect a probability of $0.5$.

For the number of cookies, we had a SE of $0.0006$, giving a margin of error of
$0.0012$ and a confidence interval of $[ 0.4988, 0.5012 ]$. The observed value
was $0.5006$, so it is considered a pass.

For the number of clicks, we had a SE of $0.0021$, giving a margin of error of
$0.0041$ and a confidence interval of $[ 0.4958, 0.5041 ]$. The observed value
was $0.5005$, so it is considered a pass.

\section{Effect Size Tests}
The measured values from the experiment are as follows:

\subsection{Gross Conversion}
\begin{itemize}
\item $P(gc|control) = 0.2189$
\item $P(gc|experiment) = 0.1983$
\item $\hat{d} = -0.0206$
\item $SE = 0.0044$
\item $ME = 8.5652 \times 10^{-3}$
\item $CI = [-0.0291,-0.0120]$
\item $d_{min} = 0.01$
\end{itemize}

Gross Conversion is therefore both statistically and practically significant. We
therefore conclude that we have observed a valid change in Gross Conversion.

\subsection{Net Conversion}
\begin{itemize}
\item $P(nc|control) = 0.1176$
\item $P(nc|experiment) = 0.1127$
\item $\hat{d} = -0.0049$
\item $SE = 3.4340 \time 10^{-3}$
\item $ME = 6.7228 \times 10^{-3}$
\item $CI = [-0.0116,0.0018]$
\item $d_{min} = 0.0075$
\end{itemize}

Net conversion is not statistically, or practically, significant. This
effectively means that we do not see a change in Net Conversions. I did not use
the Bonferroni correction, for the same reason as it was not used in the rest of
the experiment (the metrics are covariate).

\section{Sign Tests}
\subsection{Gross Conversion}
\begin{itemize}
\item $d_{min} = 0.01$
\item $N_{success} = 4$
\item $N_{total} = 23$
\item $p = 0.0026$
\item $\alpha = 0.025$
\end{itemize}

Therefore the sign test also reports significance.

\subsection{Net Conversion}
\begin{itemize}
\item $d_{min} = 0.075$
\item $N_{success} = 10$
\item $N_{total} = 10$
\item $p = 0.6776$
\item $\alpha = 0.025$
\end{itemize}

This confirms the statistical insignificance of the difference.

\section{Results Summary}
The sign test confirms the results of the effect size test. I did not use the
Bonferroni correction, because to accept or reject our null hypothesis we are
considering both Gross and Net Conversion. Therefore the logic operation between
the two sub-hypotheses is a conjunction, which means that it is not necessary to
apply the Bonferroni correction.

\section{Recommendation}
The reduction in Gross Conversions is part of what was expected of the new
change. The number of people signing up would be reduced, and there is no
significant change in the Net Conversions. This means that there are less people
signing up and then dropping out. Conversely, the lack of significance on the
Net Conversions also means that there is no significant lost (or gained)
revenue. This means that there is inherent savings in saved resources and a more
focused trial. However, the confidence interval on Net Conversions exceeds the
practical significance boundary on the negative side. As things stand, I do not
recommend launching the change, because there is still a possibility that it
would lead to a loss of revenue. Rather than completely abandoning the
experiment, I would recommend collecting more data to reduce the boundary on
this metric until its significance is clearer.

\section{Follow-up Experiment}
If the main concern is for people to be able to pass their initial trial, I
would recommend a ``mid-trial'' quiz, where the user can answer a few
questions about their expectations wrt their time commitment and if they think
they will be able to dedicate enough time to the course, and then offer them the
option to \emph{freeze} the trial for a month. The hypothesis is that the user
will be able to gauge the commitment needed for the course and reassess at a
later date.
The metrics used for this experiment would be number of users (invariant) and
retention rate (evaluation), and the unit of diversion would be the user-id,
because at this point the student is logged in and can be tracked with
precision.


\end{document}
